\documentclass[a4paper]{article}

\usepackage{amsmath}
\usepackage{amssymb}
\usepackage{graphicx}
\usepackage{polski}
\usepackage[utf8]{inputenc}

\begin{document}
\title{ \textbf{Sprawozdanie} \protect\\ Problem transportowy wielu produktów}
\author{Bartosz Wlazło, Przemysław Jekiel}
\maketitle
\pagenumbering{arabic}

\newpage 


\section{Wstęp}

\subsection{Problematyka}

Celem niniejszego badania było rozwiązanie problemu transportowego wielu produktów. 

\subsection{Przyjęty przypadek użycia}

Osoba X prowadzi sieć sklepów spożywczych. Największe koszty, które ponosi firma są związane z transportem produków z magazynów do sklepów. W celu obniżenia ich udała się ona do firmy OPTTech w celu opracowania narzędzia do planowania tras i ilości zabieranych produktów.

\section{Założenia projektowe}
\subsection{Uproszczenia}
W celu szybszej pracy kody próbka danych został ograniczona do:
\begin{itemize}
	\item 5 sklepów 
	\item 3 magazynów
	\item 4 typów produktów
	\item nieskończonej liczby środków transportu 
	\item wartości ograniczeń są wyrażone w liczbach całkowitych 
\end{itemize}

Zostało także przyjęte uproszczenie iż każdy produkt jest magazowany w tej samej liczbie.

\subsection{Definicja zmiennych}
\subsection{Szukane}
X(i,j,k) - ilość wysłanych sztuk produktu \textit{k} rodzaju do \textit{i} klienta z \textit{j} magazynu.
\subsection{Model matematyczny}

\newpage 

\section{Założenia projektowe}
\subsection{Dane ograniczeń}
\subsubsection{\textit{Pojemności magazynów}}
\begin{center}
	\begin{tabular}{ |c|c| } 
		\hline
		 Magazyn & Pojemność\\
		\hline
		Magazyn 1 & 3000  \\ 
		\hline
		Magazyn 2 & 3000  \\ 
		\hline
		Magazyn 3 & 3000  \\ 
		\hline
	\end{tabular}
\end{center}

\subsubsection{\textit{Dostępność produktów w danych magazynach}}
\begin{center}
	\begin{tabular}{ |c|c| } 
		\hline
		Magazyn & Produkt \\
		\hline
		Magazyn 1 & 2, 4  \\ 
		\hline
		Magazyn 2 & 1, 2, 3  \\ 
		\hline
		Magazyn 3 & 2, 3, 4  \\ 
		\hline
	\end{tabular}
\end{center}

\subsubsection{\textit{Zapotrzebowanie sklepów na dany produkt}}
\begin{center}
	\begin{tabular}{ |c|c|c | } 
		\hline
		Magazyn & Produkt & Zapotrzebowanie\\
		\hline
		Sklep 1 & 1 & \\ 
		\hline
		Sklep 1 & 2 & \\ 
		\hline
		Sklep 1 & 3 & \\ 
		\hline
		Sklep 1 & 4 & \\ 
		\hline
		Sklep 2 & 1 & \\ 
		\hline
		Sklep 2 & 2 & \\ 
		\hline
		Sklep 2 & 3 & \\ 
		\hline
		Sklep 2 & 4 & \\ 
		\hline
		Sklep 3 & 1 & \\ 
		\hline
		Sklep 3 & 2 & \\ 
		\hline
		Sklep 3 & 3 & \\ 
		\hline
		Sklep 3 & 4 & \\ 
		\hline
		Sklep 4 & 1 & \\ 
		\hline
		Sklep 4 & 2 & \\ 
		\hline
		Sklep 4 & 3 & \\ 
		\hline
		Sklep 4 & 4 & \\ 
		\hline
	\end{tabular}
\end{center}

\subsection{Dane}

\subsubsection{\textit{Wagi produktów}}
\begin{center}
	\begin{tabular}{ |c|c| } 
		\hline
		Produkt & Waga \\
		\hline
		Produkt 1 & 5  \\ 
		\hline
		Produkt 2 & 2  \\ 
		\hline
		Produkt 3 & 3  \\ 
		\hline
		Produkt 4 & 4  \\ 
		\hline
	\end{tabular}
\end{center}

\newpage


\section{Wynik}


\newpage 

\section{Kod}
\texttt{List.class}
\end{document}
